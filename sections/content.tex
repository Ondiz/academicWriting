% Notes for titlepage
\note{Hi everybody welcome to this course! Here we'll put into
  practice the concepts learnt in the previous theory sessions using
  LaTeX to produce academic documents.

  We're going to see how to create a document and its different parts,
  we're not going to focus on format (fontsize, colour, style\ldots)
  because when writing a paper it'll be defined by the template of the
  journal. We'll see how it works in the examples.}

\begin{frame}{What's LaTeX?}
 \begin{fullpageitemize}
  \itemR Markup language
  \itemR High quality documents
  \itemR Consistent output every time
  \itemR Separation of style from text
  \itemR Easy academic format: equations, glossaries, bibliography,
  table of contents, cross references\ldots
 \end{fullpageitemize}
 \note{LaTeX is a markup language aiming to produce high quality
   documents. It was created by academics, with the idea of having
   consistent output every time (forget that table that moved itself
   from page 2 to 5), separation of style from text (that has two
   objetives: helps centre oneself in the content and allows the reuse
   of material) and easy academic format (equations are not a pain,
   glossaries, bibliographies and tables of contents automatic and
   cross referencing trivial)
 
   So let's see how can we start using the stuff. As a note I'll talk
   mostly about free software because that's what I use}
\end{frame}

\begin{frame}{What do I need?}
 \null{\color{colororange}\largetext{Option 1}}
 \begin{itemize}
  \item \textbf{Editor}: general purpose or specific
    (\href{http://www.xm1math.net/texmaker/}{Texmaker},
    \href{https://www.tug.org/texworks/}{TeXworks},
    \href{https://kile.sourceforge.io/}{Kile})
  
  \item \textbf{Distribution}: \href{https://miktex.org/}{MikTeX}
    (Windows), \href{https://tug.org/texlive/}{TeXLive}
    (cross-platform) $\rightarrow$ compiler + packages
 \end{itemize}
 \vspace{1ex}
 {\color{colororange}\largetext{Option 2}}
  \begin{itemize}
  \item \textbf{Online editor}: \href{https://www.overleaf.com/}{Overleaf}
 \end{itemize}
 
 \note{We're going to write in plain text so any editor can do. I use
   Emacs (in general I write in Org mode and export to LaTeX). Also
   the AucTeX mode is specially useful. For starters, and with the
   idea of not getting crazy, one can use a specific editor as
   Texmaker that I used to produce these slides.
 
   We also need a distribution of LaTeX. If you're on Windows I'd
   recommend MikTeX as it takes care of needed packages on its own. If
   you're on Linux you know you to deal with packages!
 
   The other option is using an online service as Overleaf that takes
   care of the process. I use it mostly for trying templates, so if
   you plan to take this way, I can't help!}
\end{frame}

\begin{frame}{The markup language}
 \begin{fullpageitemize}
  \itemR We use labels to mark the parts of the text we want to have a
  specific style
  \itemR After compilation, we obtain a styled document
  \itemR Documents must follow a concrete structure
  \itemR Specific syntax for equations, figures, tables, titles,
  sections\ldots
\end{fullpageitemize}

 \note{So, how we write in this LaTeX stuff? Easy! We just \emph{mark}
   the parts of the text we want to have a specific format with some
   \emph{labels} that LaTeX understands, then it'll \emph{compile} our
   doc and produce a styled output.
 
   For this to be possible our documents must follow a concrete
   structure and we have to use a specific syntax for defining
   elements as equations, figures, tables or titles.}
\end{frame}

\begin{frame}[fragile]{Structure of a document}
We write \texttt{.tex} files with this structure:

 \begin{lstlisting}
   % Document definition   
   \documentclass[OPTIONS]{TYPE}
   % Preamble 
   % Data, packages and commands go here
   \begin{document}       
   % Body of document          
   \end{document}
 \end{lstlisting}
 
 \note{We write \texttt{.tex} files with a specific structure divided
   in two parts: the preamble, that goes from the definition of the
   document to the body and where we add packages, commands and data
   and after, the body of the document, where we write the contents.
 
   The cited packages are pieces of LaTeX that people wrote and give
   us added functionality as language support. They're centralised in
   a repository called \href{https://www.ctan.org/}{CTAN} and we have
   to install them to use them.
 
   In general, we'll have two types of documents: short and long, as
   we are academics, we can think of a paper and a thesis}

\end{frame}
 
\begin{frame}[fragile]{Short documents}
 \begin{lstlisting} 
 \documentclass[a4paper]{article}
  % Preamble 
  \usepackage{amsmath}
  \title{The title}
 \begin{document}       
  % Body of document  
  \maketitle
  \section*{Unnumbered section}
  \section{First numbered section}
  \subsection{First numbered subsection}
  Text!      
 \end{document}
 \end{lstlisting}
 
 \note{This is the typical structure of a short document, as a paper,
   that would probably use the article class or any of it's
   derivatives.}

\end{frame}

\begin{frame}{Short documents}
 \begin{fullpageitemize}
  \itemR We call packages in the preamble with \texttt{usepackage}
  \itemR The \texttt{article} class has 4 divisions: \texttt{section},
  \texttt{subsection}, \texttt{subsubsection} and \texttt{paragraph}
  (unnumbered)  
  \itemR The \texttt{*} suppresses numbering 
 \end{fullpageitemize} 
 
 \note{The things to take into account when writting a paper are that
   we have 4 divisions, from \texttt{section} to \texttt{paragraph},
   the first 3 are numbered by default. To suppress numbering we use
   the starred version of them, exactly the same that we'll see for
   equations.
 
   There is also here an example of loading a package in the preamble
   so that you can see the syntax.}

\end{frame}

\begin{frame}[fragile]{Long documents}
 \begin{lstlisting}
 \documentclass[a4paper]{book}
  \input{preamble} % read preamble.tex
 \begin{document}       
  \maketitle       % titlepage
  \frontmatter 
  \tableofcontents
  % Content starts here
  \mainmatter
  \include{Contents/chap1} % chap1.tex
  % Appendices start here
  \appendix 
  \include{ap1}
  \backmatter
 \end{document} 
 \end{lstlisting}
 
 \note{This is a typical long document, for example a thesis. We'll
   have a main file and call the chapters and configuration from
   outside.}

\end{frame}

\begin{frame}{Long documents}
 \begin{fullpageitemize}
  \itemR The \texttt{book} class has also \texttt{part} and
  \texttt{chapter} divisions 
  \itemR We can organise content in different files and
  \texttt{include}/\texttt{input} them
  \itemR We can mark the document (\texttt{\textbackslash
    frontmatter}, \texttt{\textbackslash mainmatter},
  \texttt{\textbackslash backmatter}) to establish different styles
  (page numbers, headings\ldots)
  \itemR The chapters after \texttt{\textbackslash appendix} are
  identified using letters
 \end{fullpageitemize}
 
 \note{When our document is long, instead of having a single gigantic
   tex file, we can divide the contents in several files and input or
   include them. The difference between including and inputing is that
   we cannot nest includes and that we can select what to compile when
   including (includeonly)
 
   We can mark the tex file with some identifiers so that the style
   varies, for example, by default, the page numbers between
   frontmatter and mainmatter are roman numerals.
 
   Then, the chapters after appendix are identified using letters.}

\end{frame}

\begin{frame}[fragile]{Defining elements}
\textbf{Titlepage}
\begin{LTXexample}[pos=r,wide,width=.48,rframe={}]
\documentclass{article}
 \title{The title}
 \author{A. Uthor}
 \date{\today}
\begin{document}
 \maketitle
\end{document}
\end{LTXexample}

\note{Let's see then how to define elements. When I previously said
  that LaTeX uncouples format and content, I meant what you can see
  here: we gave it the data in the preamble and ask it to produce a
  titlepage somewhere in the document. It takes care of everything and
  will produce an appropriate title for our document class.

  We can, of course, change the appearance of the title --- we can
  actually change everything.}

\end{frame}

\begin{frame}[fragile]{Defining elements}
\textbf{Equations}
\begin{LTXexample}[pos=r,wide,width=.3,rframe={}]
\documentclass{article}
\usepackage{amsmath, amssymb}
\begin{document}

 \begin{equation}
   A = \pi\times R^2
 \end{equation}

 % Inline equation
 where $R = 1$
\end{document}
\end{LTXexample}

\note{Here you can see the syntax for introducing equations and
  figures. The equation has a number that can be avoided using the
  equation* environment. Also, we can have both inline equations that
  are written between dollar signs. There is a special syntax for
  typesetting equations that's a bit complex to get, an online editor
  or a good IDE can help a lot at the beginning.

  I'm using the packages amsmath and amssymb to improve the appearance
  of equations and to have more commands available.}

\end{frame}

\begin{frame}[fragile]{Defining elements}
\textbf{Figures}
\begin{LTXexample}[pos=r,wide,width=.3,rframe={}]
\begin{figure}[h] % h: here
  \centering
  \includegraphics
  [width=\textwidth] % options
  {images/manatee} % path
\end{figure}
\end{LTXexample}

\begin{fullpageitemize}
 \itemR Useful package:
 \href{https://www.ctan.org/pkg/subcaption}{\texttt{subcaption}}
\end{fullpageitemize}

\note{Regarding figures, there are both inline and floating figures,
  the one shown is a floating one. We'll talk about floating elements
  after, but I'd like to point out a feature of all commands and
  environments: the options between brackets are optional and the ones
  between braces compulsory. So here, the path is of course
  compulsory. The extension can be omitted.

  The package \textbf{subcaption} allows us to include subfigures.}

\end{frame}

\begin{frame}[fragile]{Defining elements}
\textbf{Tables}
\begin{LTXexample}[pos=r,wide,width=.3,rframe={}]
\begin{table}
 \begin{tabular}{cc} % c: center
  \hline
  \textbf{A} & \textbf{B} \\ 
  \hline
  x          & 1          \\
  y          & 2          \\
  \hline
 \end{tabular}
\end{table}
\end{LTXexample}

\begin{fullpageitemize}
  \itemR Useful packages:
  \href{https://www.ctan.org/pkg/booktabs/}{\texttt{booktabs}},
  \href{https://ctan.org/pkg/longtable}{\texttt{longtable}},
  \href{https://www.ctan.org/pkg/tabularx}{\texttt{tabularx}}
\end{fullpageitemize}

\note{Here you can see an example of a table. Tables in LaTeX are a
  pain to write, so at the end I'm linking some tricks to alleviate
  the work. They can be as well floating elements or not and consist
  on some data separated by \& and backslashes.

  I added here some useful packages for tables such as
  \textbf{booktabs} that allows to create professional looking tables,
  \textbf{longtable} for tables that run across several pages and
  \textbf{tabularx} to adjust the column width of tables.}
\end{frame}

\begin{frame}[fragile]{Crossreferences}
If we add a label to an element we can reference it in the text:
\begin{LTXexample}[pos=r,wide,width=.26,rframe={}, preset=\small]
\begin{equation}
 \label{eq:im}
 \mathrm{i} = \sqrt{-1}
\end{equation}

As seen in Equation~\ref{eq:im}
\end{LTXexample}

 \begin{fullpageitemize}
  \itemR Compile two times
  \itemR Use identifiers for labels!
  \itemR Useful package:
  \href{https://ctan.org/pkg/cleveref}{\texttt{cleveref}}
 \end{fullpageitemize}

 \note{If we add a label to any element, we can after reference it in
   the text. We need to compile an additional time so as to get all
   the references. The editors let you choose (example in Texmaker). A
   good practice is to identify the references to find them with more
   ease.

   A useful package when dealing with crossreferences is
   \textbf{cleveref} that provides automatic format of reference
   depending on type.}

\end{frame}

\begin{frame}[fragile]{Captions}
Floating elements (figures, tables, listings) can have captions:

\begin{LTXexample}[pos=r,wide,width=.4,rframe={}]
\begin{table}
 \centering
 \begin{tabular}{cc}
 \hline
  x       & 1  \\
  y       & 2  \\ 
 \hline
 \end{tabular}
 \caption{Data}
\end{table}
\end{LTXexample}

\note{Floating elements as figures, tables and code listings can have
  captions. Caption is positioned where we asked it to be.}

\end{frame}

\begin{frame}[fragile]{Listing content}
Listing content is trivial:
\begin{lstlisting} 
\tableofcontents
\listoffigures
\listoftables
\end{lstlisting}

If we want to add something to toc:
\begin{lstlisting} 
\addcontentsline{toc}{chapter}{NAME}
\end{lstlisting}

We can define short names for sections and captions for the lof/lot:
\begin{lstlisting} 
\caption[SHORT]{LONG}
\end{lstlisting}

\note{Making a table of contents, list of figures or tables is
  trivial, we just add the command where we want it placed and done!
  We have to take into account two things: unnumbered chapters are not
  added to the table of contents, sections and captions have the
  option of a short name for the lists.}

\end{frame}

\begin{frame}{Bibliography}
 \begin{fullpageitemize}
  \itemR Bibliography stored in a \texttt{.bib} file 
  \itemR Managed by external program (\href{http://www.bibtex.org/}{BibTeX} \ldots) 
  \itemR Additional compilation step: \texttt{(pdf)latex - bibtex - (pdf)latex - (pdf)latex}
  \itemR Bibliography managers: JabRef, Zotero, \href{https://bibdesk.sourceforge.io/}{BibDesk}\ldots
 \end{fullpageitemize}
 
 \note{The most important part for academics like us is the ability to
   cite easily. In LaTeX citation is managed by an external program
   called BibTeX (there are more "advanced" options like Natbib,
   BibLaTeX) that comes with the distribution and requires and
   additional compilation step. So, if we have both crossreferences
   and bibliographic references, we'll have to compile 4 times in a
   sequence \texttt{(pdf)latex - bibtex - (pdf)latex - (pdf)latex}.
 
   We need a \texttt{.bib} file that can be exported from our
   bibliography manager. For example, I use JabRef that works directly
   with \texttt{.bib} files.}
 
\end{frame}

\begin{frame}[fragile]{Bibliography}
MWE with bibliography defined in \texttt{references.bib}:

\begin{lstlisting}
\documentclass[11pt]{article}
  \begin{document}
  
  As said in \cite{Knuth1997}
    
  % Definition of bibliography
  \bibliographystyle{plain} % Style
  \bibliography{references} % Path
\end{document}
\end{lstlisting}

 \begin{fullpageitemize}
  \itemR Collection of styles in \href{http://debibify.dorian-depriester.fr/}{Debibify}
 \end{fullpageitemize}

 \note{So, a minimum working example for a document with
   bibliographies would be something like this. The
   \texttt{\textbackslash cite} command introduces the cite into the
   text in the style defined. The \texttt{\textbackslash bibliography}
   writes down the references, alphabetically or in order of
   appearance depending on the style. We can have as many references
   as we want in the \texttt{.bib} file, LaTeX is smart and will only
   add the ones we have cited.}

\end{frame}

\begin{frame}[fragile]{Bibliography}
Where \texttt{references.bib} looks like:

\begin{lstlisting}[breaklines=false]
@book{Knuth1997,
  title={The art of computer programming},
  author={Knuth, Donald Ervin},
  volume={3},
  year={1997},
  publisher={Pearson Education}}
\end{lstlisting}

\vspace{3ex}
 \begin{fullpageitemize}
  \itemR \texttt{Knuth1997} is the \textbf{key} that we use for citing
 \end{fullpageitemize}

 \note{The \texttt{.bib} file has entries that look like this one,
   where the first word is the key that we'll use for citing. A little
   trick is changing Scholar settings so that one can directly import
   to BibTeX (EXAMPLE)}

\end{frame}

\begin{frame}{Useful packages}
 \begin{fullpageitemize}
  \itemR \href{https://ctan.org/tex-archive/macros/latex/contrib/units/}{\texttt{units}}: typesets units properly
  \itemR \href{https://www.ctan.org/pkg/listofsymbols}{\texttt{listofsymbols}}: builds automatic list of symbols
  \itemR \href{https://ctan.org/tex-archive/macros/latex/contrib/microtype/}{\texttt{microtype}}: deals with microtypography
  \itemR \href{https://www.ctan.org/pkg/listings}{\texttt{listings}}: provides syntax highlighting for code
\end{fullpageitemize}

\note{I'm sharing here some packages I find useful, in the practice
  time take a look and ask me! The first three help dealing with
  references, units and symbols respectively and the last two are for
  improving the appearance of the document. (Example)}

\end{frame}

\begin{frame}{Tools \& tricks}
 \begin{fullpageitemize}
  \itemR \href{https://www.overleaf.com/}{Overleaf}: templates and online editor
  \itemR \href{https://www.ctan.org/pkg/latexmk/}{latexmk}: automate document generation
  \itemR \href{http://wiki.inkscape.org/wiki/index.php/LaTeX}{Inkscape}: image text as document text, LaTeX equation in images
  \itemR \href{https://www.ctan.org/tex-archive/support/excel2latex/}{Excel2LaTeX}: export tables with LaTeX format
  \itemR \href{http://detexify.kirelabs.org/classify.html}{Detexify}: find symbols
  \itemR \href{http://pandoc.org/}{Pandoc}: convert to any format
  \itemR \href{https://www.lyx.org/}{LyX}: middle point between Writer and LaTeX
\end{fullpageitemize}
 
\note{\footnotesize I'm sharing here some tools and tricks. We've
  talked about Overleaf before, I use it mainly for searching
  template. Latexmk is a Perl script to automate the document
  generation. Then we have Inkscape that lets us export the text as
  LaTeX so that the font is the same as the one in the
  document. Tables are a pain, there is this Excel plugin to export
  them from Excel. Then my two favourite tools: Detexify, that finds a
  symbol that we write, and Pandoc, a command line tool to convert
  from any format to another so we can always write in LaTeX. Finally,
  we have LyX, that's an editor similar to Word but that uses LaTeX in
  the background, great to learn!}
\end{frame}

% Testing: \href{http://mathb.in}{MathB.in}; \href{https://mathpix.com/}{Mathpix}: screenshot to equation

\begin{frame}{Case study: article}
 
 \begin{fullpageitemize}
	\itemR Select journal 
	\itemR Find LaTeX template or Guide for authors
	\itemR Find example or docs of LaTeX class
 \end{fullpageitemize} 
 \vspace{2ex}
 {\color{colororange}\largetext{Example}}
 \vspace{2ex}
 \begin{fullpageitemize}
   \itemR
   \href{https://www.journals.elsevier.com/engineering-structures/}{Engineering
     structures} $\rightarrow$
   \href{https://ctan.org/tex-archive/macros/latex/contrib/elsarticle}{\texttt{elsarticle.cls}}
   \itemR
   \href{https://www.elsevier.com/journals/engineering-structures/0141-0296/guide-for-authors}{Guide
     for authors}
   \itemR \href{https://ctan.org/tex-archive/macros/latex/contrib/elsarticle/doc}{Example of use}
 \end{fullpageitemize} 
 
 \note{We have then two case studies: a paper and a thesis. For a
   paper we have to choose first where to send it and find the LaTeX
   template (EXAMPLE)}
\end{frame}

\begin{frame}{Case study: thesis}
 
 \begin{fullpageitemize}
    \itemR Choose a template $\rightarrow$ \href{https://www.overleaf.com/}{Overleaf}
    \itemR Proposed file system: 
    \vspace{1ex}
    \begin{fullpageitemize}
     \item[-] Main \texttt{.tex}
     \item[-] Bibliography in \texttt{.bib}
     \item[-] \texttt{Contents} folder $\rightarrow$ \texttt{.tex} files for chapters
     \item[-] \texttt{Figures} folder $\rightarrow$ subfolders for chapters
    \end{fullpageitemize}
    \itemR Consider \href{https://git-scm.com/book/en/v2}{version control}
    \itemR If not writing in English $\rightarrow$ \href{https://en.wikibooks.org/wiki/LaTeX/Internationalization}{language support}
 \end{fullpageitemize} 
 
 \note{The second case we're going to consider is a thesis. In this
   case organisation is a must! Having a logic file system and
   thinking in advance how are we going to treat symbols, units and
   the like is half the work. I propose something like this, with a
   main \texttt{.tex} and different ones for chapters inside a
   folder. And please use version control, I can't stress this more,
   please do it!
 
   As a note, if you're not writing in English, you must set the
   language, it's not trivial but takes care of the name of elements,
   the date and other stuff on its own.}
 
\end{frame}

\begin{frame}{References}
 \begin{fullpageitemize}
    \itemR\href{https://ondiz.github.io/cursoLatex/}{Curso no convencional de LaTeX}. Ondiz Zarraga
    \itemR \href{http://www.khirevich.com/latex/}{Tips on Writing a Thesis in LaTeX}. Siarhei Khirevich
    \itemR \href{https://www.overleaf.com/learn/latex/Main_Page}{Overleaf Docs}
 \end{fullpageitemize}
 
 \note{To finish here you have a couple of references. The first is a
   online book I wrote when I was on the dole and it's in Spanish. And
   remember, Stackoverflow is your friend! A neat trick is to search
   using DuckDuckGo that provides cheatsheets and direct answers from
   StackOverflow.
 
   Let's go!}
\end{frame}

\framecard[colorgreen]{
\vspace*{3ex}
{\color{white}\hugetext{Let's go!}}
% xkcd
\begin{figure}[h]
  \includegraphics[width=0.4\textwidth]{images/file_extensions.png}
\end{figure}
{\tiny\href{https://xkcd.com/1301/}{\textcolor{white}{https://xkcd.com/1301/}}}
}
%%% Local Variables:
%%% TeX-master: "../academicWriting"
%%% TeX-engine: xetex
%%% End:
