% !TeX root = ..\academicWriting.tex SHOULD WORK BUT DOESN'T :(

\begin{frame}{What's LaTeX?}
 \begin{fullpageitemize}
  \itemR Markup language
  \itemR Easy academic format: glossaries, bibliography, table of contents, cross references\ldots
  \itemR Equations
 \end{fullpageitemize}
 \note{I'm talking only about free software because that's what I use}
\end{frame}

\begin{frame}{What do I need?}
 \null{\color{colororange}\largetext{Option 1}}
 \begin{itemize}
  \item \textbf{Editor}: general purpose or specific (\href{http://www.xm1math.net/texmaker/}{Texmaker}, \href{https://www.tug.org/texworks/}{TeXworks}, \href{https://kile.sourceforge.io/}{Kile})
  
  \item \textbf{Distribution}: \href{https://miktex.org/}{MikTeX} (Windows), \href{https://tug.org/texlive/}{TeXLive} (cross-platform) $\rightarrow$ compiler + packages
 \end{itemize}
 \vspace{1ex}
 {\color{colororange}\largetext{Option 2}}
  \begin{itemize}
  \item \textbf{Online editor}: \href{https://www.overleaf.com/}{Overleaf}
 \end{itemize}
 
 \note{Any editor can do, I use Emacs (in general I write in Org mode and export to LaTeX). The AucTeX mode is specially useful.
 
 LyX
 
 If you're on Windows I'd recommend MikTeX as it takes care of needed packages on its own.}
\end{frame}

\begin{frame}{The markup language}
- Packages, preamble, environments and commands
- Equations, tables, titles, figures 
\end{frame}

\begin{frame}{Structure of a document}

\end{frame}

\begin{frame}{Useful packages}

\end{frame}

\begin{frame}{Bibliography}
 \begin{fullpageitemize}
  \itemR Bibliography stored in a {\inconsolatafont .bib} file 
  \itemR Managed by external program (\href{http://www.bibtex.org/}{BibTeX}, \ldots) 
  \itemR Additional compilation step: {\inconsolatafont latex - bibtex - latex - latex}
  \itemR Bibliography managers: JabRef, Zotero, \href{https://bibdesk.sourceforge.io/}{BibDesk} \ldots
 \end{fullpageitemize}
 
 \note{zotero-better-bibtex,  for Mac. 
 BibTeX comes with the standard LaTeX distribution, there are more options like Natbib, BibLaTeX}
\end{frame}

\begin{frame}[fragile]{Bibliography}

MWE with bibliography defined in {\inconsolatafont references.bib}:

\begin{lstlisting}[language={[LaTeX]TeX},texcsstyle=*\color{colororange}]
\documentclass[11pt]{article}
  \begin{document}
  
  As said in \cite{Knuth1997}...
    
  % Definition of bibliography
  \bibliography{references} % Path
  \bibliographystyle{plain} % Style
\end{document}
\end{lstlisting}

\note{Minimum working example}

\end{frame}

\begin{frame}[fragile]{Bibliography}

Where {\inconsolatafont references.bib} looks like:

\begin{lstlisting}[breaklines=false]
@book{Knuth1997,
  title={The art of computer programming},
  author={Knuth, Donald Ervin},
  volume={3},
  year={1997},
  publisher={Pearson Education}}
\end{lstlisting}
\note{A little trick is changing Scholar settings so that one can directly import to BibTex}
\end{frame}

\begin{frame}{Tools \& tricks}
 \begin{fullpageitemize}
  \itemR \href{https://www.overleaf.com/}{\textbf{Overleaf}}: templates and online editor
  \itemR \href{http://wiki.inkscape.org/wiki/index.php/LaTeX}{\textbf{Inkscape}}: image text as document text, LaTeX equation in images
  \itemR \href{https://www.ctan.org/tex-archive/support/excel2latex/}{\textbf{Excel2LaTeX}}: export tables with LaTeX format
  \itemR \href{http://detexify.kirelabs.org/classify.html}{\textbf{Detexify}}: find symbols
  \itemR \href{http://pandoc.org/}{\textbf{Pandoc}}: convert to \emph{any} format
 \end{fullpageitemize}
 
\note{\href{http://mathb.in}{MathB.in}} 
\end{frame}

\begin{frame}{References}
 \begin{fullpageitemize}
	\itemR\href{https://ondiz.github.io/cursoLatex/}{\emph{Curso no convencional de LaTeX}}. Ondiz Zarraga

	\itemR \href{http://www.khirevich.com/latex/}{\emph{Tips on Writing a Thesis in LaTeX}}. Siarhei Khirevich
 \end{fullpageitemize}
\end{frame}

\framecard[colorgreen]{
{\color{white}\hugetext{Let's go!}}
% xkcd
}